\section{Experimental Evaluation}
\label{sec:experiments}

\paragraph{Dataset} We test our DeepLab model on the PASCAL VOC 2012 segmentation benchmark \citep{everingham2014pascal}, consisting of 20 foreground object classes and one background class. The original dataset contains $1,464$, $1,449$, and $1,456$ images for training, validation, and testing, respectively. The dataset is augmented by the extra annotations provided by \citet{hariharan2011semantic}, resulting in $10,582$ training images. The performance is measured in terms of pixel intersection-over-union (IOU) averaged across the 21 classes. 

\paragraph{Training} We adopt the simplest form of piecewise training, decoupling the DCNN and CRF training stages, assuming the unary terms provided by the DCNN are fixed during CRF training. 

For DCNN training we employ the VGG-16 netwok which has been pre-trained on ImageNet. We fine-tuned the VGG-16 network on the VOC 21-way classification task by stochastic gradient descent on the cross-entropy loss function, as described in Section~\ref{sec:convnet-hole}. We use a mini-batch of 20 images and initial learning rate of $0.001$ ($0.01$ for the final classifier layer), multiplying the learning rate by 0.1 at every 2000 iterations. We use momentum of $0.9$ and a weight decay of $0.0005$.

%a DCNN is first fine-tuned on the training set and then, as in the work of \citet{krahenbuhl2011efficient}  we cross-validate the 
After the DCNN has been fine-tuned, we cross-validate the parameters of the fully connected CRF model in \equref{eq:fully_crf} along the lines of \citet{krahenbuhl2011efficient};
 to avoid overfitting the validation set, the parameters $w_2$ and $\sigma_\gamma$ are fixed to be $3$ and the best values of $w_1$, $\sigma_\alpha$, and $\sigma_\beta$ are cross-validated on a small subset of the validation set (we use 200 images). We use 10 iterations for the efficient mean field inference algorithm \citep{krahenbuhl2011efficient}. 

\paragraph{Evaluation on Validation set} We conduct the majority of our evaluations on the PASCAL `val' set, training our model on the augmented PASCAL `train' set. As shown in \tabref{tb:valIOU}, incorporating the fully connected CRF to our model (denoted by DeepLab-CRF) yields a substantial performance boost, about 4\% improvement. We note that the original work of \citet{krahenbuhl2011efficient} improved the $27.6\%$
result of TextonBoost \citep{shotton2009textonboost} to $29.1\%$, which makes the  improvement we report here (from $59.8\%$ to $63.7\%$) all the more impressive.

Turning to qualitative results, we provide visual comparisons between DeepLab and DeepLab-CRF in \figref{fig:ValResults}. Employing a fully connected CRF significantly improves the results, allowing the model to accurately capture intricate object boundaries. 

\begin{table}
  \centering
  \begin{tabular}{c c}
    \raisebox{0.4cm}{
    \begin{tabular}{c | c}
      Method      & mean IOU (\%) \\
      \hline \hline
      DeepLab     & 59.80 \\
      DeepLab-CRF & 63.74
    \end{tabular}
    }
    &
    \begin{tabular}{c | c}
      Method      & mean IOU (\%) \\
      \hline \hline
      MSRA-CFM    & 61.8 \\
      FCN-8s      & 62.2 \\
      TTI-Zoomout-16 & 64.4 \\
      \hline
      DeepLab-CRF (our) & 66.4
    \end{tabular}
    \\
    (a) & (b)
  \end{tabular}
  \caption{(a) Performance of our proposed models on the PASCAL VOC 2012 'val' set (with training in the augmented 'train' set). (b) Performance of our best DeepLab-CRF model compared to other state-of-art methods on the PASCAL VOC 2012 'test' set.}
  \label{tb:valIOU}
\end{table}


%{\bf{Weighted loss: }} In PASCAL VOC 2012 dataset, most of the pixels are labeled as background in the ground truths. Weighting the loss function according to the class frequency has been employed, \eg \citet{farabet2013learning, mostajabi2014feedforward}, to overcome the label imbalance problem. Our experiments on VOC 2012 val set show that using a weighted loss function increases the mean class accuracy (the pixelwise accuracy averaged across classes), but does not improve the mean IOU in our model. 

\paragraph{Mean Pixel IOU along Object Boundaries}
To quantify the accuracy of the proposed model near object boundaries, we evaluate the segmentation accuracy with an experiment similar to \citet{kohli2009robust, krahenbuhl2011efficient}. Specifically, we  use  the `void' label annotated in val set, which usually occurs around object boundaries. We compute the mean IOU for those pixels that are located within a narrow band (called trimap) of `void' labels. As shown in \figref{fig:IOUBoundary}, refining the segmentation results by a fully connected CRF significantly improves the results around object boundaries. 

\paragraph{Comparison with State-of-art} In \figref{fig:val_comparison}, we qualitatively compare our proposed model, DeepLab-CRF, with two state-of-art models: FCN-8s \citep{long2014fully} and TTI-Zoomout-16 \citep{mostajabi2014feedforward} on the `val' set (the results are extracted from their papers). Our model is able to capture the intricate object boundaries.

\begin{figure}[h]
  \centering
  \begin{tabular}{c c}
    \includegraphics[height=0.55\linewidth]{fig/comparedWithFCN.pdf} &
    \includegraphics[height=0.55\linewidth]{fig/comparedWithRoomOut.pdf} \\
    (a) FCN-8s vs. DeepLab-CRF & (b) TTI-Zoomout-16 vs. DeepLab-CRF \\
  \end{tabular}
  \caption{Comparisons with state-of-the-ard models on the val set. First row: images. Second row: ground truths. Third row: other recent models (Left: FCN-8s, Right: TTI-Zoomout-16). Fourth row: our DeepLab-CRF.}
  \label{fig:val_comparison}
\end{figure}


\paragraph{Test set results} Having set our model choices on the validation set, we evaluate our best model variant, DeepLab-CRF, on the PASCAL VOC 2012 official 'test' set.  As shown in \tabref{tab:voc2012}, our model achieves performance of $66.4\%$ mean IOU\footnote{\url{http://host.robots.ox.ac.uk:8080/leaderboard/displaylb.php?challengeid=11&compid=6}}, outperforming all the other state-of-the-art models (specifically, TTI-Zoomout-16 \citep{mostajabi2014feedforward}, FCN-8s \citep{long2014fully}, and MSRA-CFM \citep{dai2014convolutional}).

\begin{table*}[ht]\scriptsize
\setlength{\tabcolsep}{3pt}
%\hspace{-1.8cm}
\resizebox{\columnwidth}{!}{
\begin{tabular}{|c||c*{20}{|c}||c|}
\hline 
Method         & bkg &  aero & bike & bird & boat & bottle& bus & car  &  cat & chair& cow  &table & dog  & horse & mbike& person& plant&sheep& sofa &train & tv   & mean \\
\hline \hline
MSRA-CFM       & -    & 75.7 & 26.7 & 69.5 & 48.8 & {\bf 65.6} & 81.0 & 69.2 & 73.3 & {\bf 30.0} & 68.7 & 51.5 & 69.1 & 68.1  & 71.7 & 67.5 & 50.4 & 66.5 & 44.4 & 58.9 & 53.5 & 61.8 \\
FCN-8s         & -    & 76.8 & 34.2 & 68.9 & 49.4 & 60.3 & 75.3 & 74.7 & 77.6 & 21.4 & 62.5 & 46.8 & 71.8 & 63.9  & 76.5 & 73.9 & 45.2 & 72.4 & 37.4 & 70.9 & 55.1 & 62.2 \\
TTI-Zoomout-16 & 89.8 & {\bf 81.9} & {\bf 35.1} & {\bf 78.2} & {\bf 57.4} & 56.5 & 80.5 & 74.0 & {\bf 79.8} & 22.4 & {\bf 69.6} & {\bf 53.7} & 74.0 & {\bf 76.0} & 76.6 & 68.8 & 44.3 & 70.2 & 40.2 & 68.9 & 55.3 & 64.4 \\
\hline
DeepLab-CRF    &{\bf 92.1} & 78.4 & 33.1 & {\bf 78.2} & 55.6 & 65.3 & {\bf 81.3} & {\bf 75.5} & 78.6 & 25.3 & 69.2 & 52.7 & {\bf 75.2} & 69.0  & {\bf 79.1} & {\bf 77.6} & {\bf 54.7} & {\bf 78.3} & {\bf 45.1} & {\bf 73.3} & {\bf 56.2} & {\bf 66.4} \\ 
\hline
 \end{tabular}
}
 \caption{Labeling IoU (\%) on the PASCAL VOC 2012 test set, using the trainval set for training.}
 \label{tab:voc2012}
\end{table*}



% \begin{figure}[ht]
%   \centering
%   \begin{tabular}{c c c | c c}
%     \includegraphics[height=0.12\linewidth]{fig/img/2007_002266.jpg} &
%     \includegraphics[height=0.12\linewidth]{fig/fcn8s/2007_002266.png} &
%     \includegraphics[height=0.12\linewidth]{fig/res_crf/2007_002266.png} &
%     \includegraphics[height=0.12\linewidth]{fig/SegPixelAccWithinTrimap_Berkeley.pdf} &
%     \includegraphics[height=0.12\linewidth]{fig/SegPixelIOUWithinTrimap_Berkeley.pdf} \\
%     (a) & (b) & (c) & (d) & (e)
%   \end{tabular}
%   \caption{(Left) Some comparisons with FCN-8S: (a) image; (b) FCN-8S; (c)
%     ms-crf. (d) Segmentation accuracy (pixelwise accuracy) within trimap. (e)
%     Segmentation accuracy (mean IOU) within trimap. {\color{red} TODO: change
%       legend. HELP: I cannot make them equally spaced....}} 
%   \label{fig:IOUBoundary}
% \end{figure}

\begin{figure}
\centering
\resizebox{\columnwidth}{!}{
  \begin{tabular} {c c c}
%    \hspace{-0.5cm}\raisebox{2cm}
    \raisebox{1.7cm} {
    \begin{tabular}{c c}
      \includegraphics[height=0.1\linewidth]{fig/trimap/2007_000363.jpg} &
      \includegraphics[height=0.1\linewidth]{fig/trimap/2007_000363.png} \\
      \includegraphics[height=0.1\linewidth]{fig/trimap/TrimapWidth2.pdf} &
      \includegraphics[height=0.1\linewidth]{fig/trimap/TrimapWidth10.pdf} \\
    \end{tabular} } &
    \includegraphics[height=0.25\linewidth]{fig/SegPixelAccWithinTrimap.pdf} &
    \includegraphics[height=0.25\linewidth]{fig/SegPixelIOUWithinTrimap.pdf} \\
    (a) & (b) & (c) \\
   \end{tabular}
}
  \caption{(a) Some trimap examples (top-left: image. top-right: ground-truth. bottom-left: trimap of 2 pixels. bottom-right: trimap of 10 pixels). Quality of segmentation result within a band around the object boundaries for the proposed methods. (b) Pixelwise accuracy. (c) Pixel mean IOU. 
    }  
  \label{fig:IOUBoundary}
\end{figure}


\begin{figure}[!htbp]
  \centering
  %\vspace{-1.cm}
  \scalebox{0.82} {
  \begin{tabular}{c c c | c c c}
    %\addtolength{\tabcolsep}{-6.5pt}
    \includegraphics[height=0.12\linewidth]{fig/img/2007_002094.jpg} &
    \includegraphics[height=0.12\linewidth]{fig/res_none/2007_002094.png} &
    \includegraphics[height=0.12\linewidth]{fig/res_crf/2007_002094.png} &
    \includegraphics[height=0.12\linewidth]{fig/img/2007_002719.jpg} &
    \includegraphics[height=0.12\linewidth]{fig/res_none/2007_002719.png} &
    \includegraphics[height=0.12\linewidth]{fig/res_crf/2007_002719.png} \\
    \includegraphics[height=0.12\linewidth]{fig/img/2007_003957.jpg} &
    \includegraphics[height=0.12\linewidth]{fig/res_none/2007_003957.png} &
    \includegraphics[height=0.12\linewidth]{fig/res_crf/2007_003957.png} &
    \includegraphics[height=0.12\linewidth]{fig/img/2007_003991.jpg} &
    \includegraphics[height=0.12\linewidth]{fig/res_none/2007_003991.png} &
    \includegraphics[height=0.12\linewidth]{fig/res_crf/2007_003991.png} \\
    \includegraphics[height=0.10\linewidth]{fig/img/2008_001439.jpg} &
    \includegraphics[height=0.10\linewidth]{fig/res_none/2008_001439.png} &
    \includegraphics[height=0.10\linewidth]{fig/res_crf/2008_001439.png} &
    \includegraphics[height=0.12\linewidth]{fig/img/2008_004363.jpg} &
    \includegraphics[height=0.12\linewidth]{fig/res_none/2008_004363.png} &
    \includegraphics[height=0.12\linewidth]{fig/res_crf/2008_004363.png} \\
    \includegraphics[height=0.12\linewidth]{fig/img/2008_006229.jpg} &
    \includegraphics[height=0.12\linewidth]{fig/res_none/2008_006229.png} &
    \includegraphics[height=0.12\linewidth]{fig/res_crf/2008_006229.png} &
    \includegraphics[height=0.12\linewidth]{fig/img/2009_000412.jpg} &
    \includegraphics[height=0.12\linewidth]{fig/res_none/2009_000412.png} &
    \includegraphics[height=0.12\linewidth]{fig/res_crf/2009_000412.png} \\
    \includegraphics[height=0.12\linewidth]{fig/img/2009_000421.jpg} &
    \includegraphics[height=0.12\linewidth]{fig/res_none/2009_000421.png} &
    \includegraphics[height=0.12\linewidth]{fig/res_crf/2009_000421.png} &
    \includegraphics[height=0.12\linewidth]{fig/img/2010_001079.jpg} &
    \includegraphics[height=0.12\linewidth]{fig/res_none/2010_001079.png} &
    \includegraphics[height=0.12\linewidth]{fig/res_crf/2010_001079.png} \\
    \includegraphics[height=0.12\linewidth]{fig/img/2010_000038.jpg} &
    \includegraphics[height=0.12\linewidth]{fig/res_none/2010_000038.png} &
    \includegraphics[height=0.12\linewidth]{fig/res_crf/2010_000038.png} &
    \includegraphics[height=0.12\linewidth]{fig/img/2010_001024.jpg} &
    \includegraphics[height=0.12\linewidth]{fig/res_none/2010_001024.png} &
    \includegraphics[height=0.12\linewidth]{fig/res_crf/2010_001024.png} \\
    \includegraphics[height=0.24\linewidth]{fig/img/2007_005331.jpg} &
    \includegraphics[height=0.24\linewidth]{fig/res_none/2007_005331.png} &
    \includegraphics[height=0.24\linewidth]{fig/res_crf/2007_005331.png} &
    \includegraphics[height=0.24\linewidth]{fig/img/2008_004654.jpg} &
    \includegraphics[height=0.24\linewidth]{fig/res_none/2008_004654.png} &
    \includegraphics[height=0.24\linewidth]{fig/res_crf/2008_004654.png} \\
    \includegraphics[height=0.24\linewidth]{fig/img/2007_000129.jpg} &
    \includegraphics[height=0.24\linewidth]{fig/res_none/2007_000129.png} &
    \includegraphics[height=0.24\linewidth]{fig/res_crf/2007_000129.png} &
    \includegraphics[height=0.24\linewidth]{fig/img/2007_002619.jpg} &
    \includegraphics[height=0.24\linewidth]{fig/res_none/2007_002619.png} &
    \includegraphics[height=0.24\linewidth]{fig/res_crf/2007_002619.png} \\
    \includegraphics[height=0.12\linewidth]{fig/img/2007_002852.jpg} &
    \includegraphics[height=0.12\linewidth]{fig/res_none/2007_002852.png} &
    \includegraphics[height=0.12\linewidth]{fig/res_crf/2007_002852.png} &
    \includegraphics[height=0.12\linewidth]{fig/img/2010_001069.jpg} &
    \includegraphics[height=0.12\linewidth]{fig/res_none/2010_001069.png} &
    \includegraphics[height=0.12\linewidth]{fig/res_crf/2010_001069.png} \\
    \hline
    \hline
    \includegraphics[height=0.12\linewidth]{fig/img/2007_000491.jpg} &
    \includegraphics[height=0.12\linewidth]{fig/res_none/2007_000491.png} &
    \includegraphics[height=0.12\linewidth]{fig/res_crf/2007_000491.png} &
    \includegraphics[height=0.12\linewidth]{fig/img/2007_000529.jpg} &
    \includegraphics[height=0.12\linewidth]{fig/res_none/2007_000529.png} &
    \includegraphics[height=0.12\linewidth]{fig/res_crf/2007_000529.png} \\
    \includegraphics[height=0.12\linewidth]{fig/img/2007_000559.jpg} &
    \includegraphics[height=0.12\linewidth]{fig/res_none/2007_000559.png} &
    \includegraphics[height=0.12\linewidth]{fig/res_crf/2007_000559.png} &
    \includegraphics[height=0.12\linewidth]{fig/img/2007_000663.jpg} &
    \includegraphics[height=0.12\linewidth]{fig/res_none/2007_000663.png} &
    \includegraphics[height=0.12\linewidth]{fig/res_crf/2007_000663.png} \\    
    \includegraphics[height=0.12\linewidth]{fig/img/2007_000452.jpg} &
    \includegraphics[height=0.12\linewidth]{fig/res_none/2007_000452.png} &
    \includegraphics[height=0.12\linewidth]{fig/res_crf/2007_000452.png} &
    \includegraphics[height=0.12\linewidth]{fig/img/2007_002268.jpg} &
    \includegraphics[height=0.12\linewidth]{fig/res_none/2007_002268.png} &
    \includegraphics[height=0.12\linewidth]{fig/res_crf/2007_002268.png} \\
  \end{tabular}
  }
  %\vspace{-0.3cm}
  \caption{Visualization results on VOC 2012-val. For each row, we show the input image, the segmentation result delivered by the DCNN (DeepLab), and the refined segmentation result of the Fully Connected CRF (DeepLab-CRF). We show our failure modes in the last three rows.} 
  \label{fig:ValResults}
\end{figure}
